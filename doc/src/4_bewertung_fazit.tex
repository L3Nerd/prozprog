\chapter{Bewertung und Fazit}

Wir möchten zum Schluss der Dokumentation noch eine Bewertung der Aufgabe und des Ablaufes ziehen ehe wir mit einem kurzen persönlichem Fazit enden.

\section{Bewertung}

Der Ablauf des Praktikums war im Gegensatz zu den anderen Praktika (Computergrafik, Rechnernetze) terminlich ungewöhnlich gelegt. Wir glauben aber, dass das kein negativer Kritikpunkt ist, denn so kann man sich als Student während des Semesters besser auf andere Übungen konzentrieren, was die Überschneidung von Fristen und den damit oft verbundenen Stress verringert. Natürlich kann es auch so Konflikten mit möglichen Urlaubsplanungen nach den Klausuren kommen, was bei uns in aktuellen Semester aber nicht der Fall war. In diesem Fall hatte man ja auch die Möglichkeit, das Praktikum vor dieser Phase abzuschließen.

Aus mehreren Gründen fanden wir die Aufgabenstellung des Praktikums gut gewählt:\\
Zum Ersten war die Aufgabe nicht ganz so umfangreich wie von uns zuvor, durch das Rechnernetze Praktikum geprägt, befürchtet wurde. Allerdings hatten wir weitestgehend auch keine Schwierigkeiten, da wir beide zuvor das Praktikum Computergrafiken absolvierten und somit noch gut mit der Sprachen vertraut waren, was die benötigte Zeit wahrscheinlich in Grenzen gehalten hat.\\
Zudem konnte man sich unter der Aufgabe etwas vorstellen, und alleine schon durch die Benennung der Funktionen mit Namen wie \textit{placeWurst} war die Grundstimmung während des gesamten Praktikums eine positivere als sie bei einer langweiligeren Aufgabenstellung gewesen wäre.\\
Letztendlich war die Aufgabe auch deutlich gestellt, und ließ trotzdem an vielen Stellen gewollt Raum für Eigeninterpretationen. Auch waren die User-Manuals, obwohl teilweise fehlerhaft und doch sehr umfangreich, nach einer recht kurzen Eingewöhnung ein sehr gutes Hilfsmittel. Durch diese Kombination kam die Notwendigkeit von Rückfragen meist erst gar nicht auf.

\section{Fazit}

Wir haben gemerkt, dass wir durch die praktische Anwendung der theoretischen Inhalte aus der Vorlesung \textit{Echtzeitsysteme} die Zusammenhänge und tatsächlichen Anwendungsmöglichkeiten der verschiedenen Kommunikationsdienste gut vertiefen konnten. Wir lernten außerdem wie aufwendig und schwierig selbst das Testen von vergleichsweise einfachen Echtzeitsystemen ist. Dadurch schätzt man die Arbeit an harten Echtzeitsystemen in der Wirtschaft umso mehr.\\
Insgesamt hat das Praktikum uns beiden Spaß gemacht; naja - so viel Spaß so ein Praktikum halt machen kann.