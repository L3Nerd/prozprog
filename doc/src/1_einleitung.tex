\chapter{Einleitung}\label{ch:einleitung}

In diesem ersten Kapitel dieser Arbeit soll zunächst die Motivation für ihre Erstellung dargestellt werden. Nachdem dann ihr grober Aufbau erklärt wird, folgt noch eine allgemeine Beschreibung der Aufgabe, die einen Überblick über ihre Problemstellung geben soll.

\section{Motivation}

Die Motivation dieser Arbeit war die Erfüllung der Veranstaltung \textit{Praktikum Echtzeitsysteme} für zwei ECTS Punkte. In diesem Semester sollten im Rahmen des Praktikums zum einen mithilfe verschiedener Kommunikationsdienste als echtzeitfähiges Programm zu einer gegebenen Problemstellung realisiert werden. Die zweite Teilaufgabe war diese ergänzende Dokumentation zum Ablauf der Bearbeitung der Aufgabe.

\section{Aufbau der Dokumentation}

In dieser Dokumentation werden die Prozesse der verschiedenen Entwicklungsphasen für die Aufgabenstellung beschrieben. Nach der Einführung, folgen angefangen mit der Darstellung und Analyse der Aufgabenstellung die Planung der einzelnen Tasks und Datenstrukturen. Anschließend wird noch beschrieben, wie die Implementierung des entstandenen Konzepts realisiert wurde, ergänzt durch eine Übersicht der Tests die hierfür erstellt und genutzt wurden. Zuletzt enden wir mit einer persönliche Bewertung der Aufgabe und der Abläufe in unserer Gruppe, sowie einem kurzen Fazit zum gesamten Praktikum.

\section{Allgemeine Problemstellung}

Dieses Semester wurde die Aufgabe \textit{Grillfest} gestellt. Es war eine echtzeitfähige Simulation des Grillprozesses zu realisieren. Der Prozess sollten vom Erstellen der Würste, über das Legen auf den Grill und dem Bräunen, bis zum Verzehr dargestellt werden.\\
Der Ablauf sollte über verschiedene Akteure funktionieren, die entweder physikalischer oder persönlicher Natur sein konnten und vom Benutzer mehr oder weniger zu beeinflussen waren.\\
Eine genauere Erklärung der Problemstellung folgt im nächsten Kapitel.