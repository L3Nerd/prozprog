\chapter{Implementierung}

Das Kapitel Implementierung beinhaltet neben der Darstellung der Abläufe unserer Gruppe während der Realisierung des Konzepts, das aus der Analyse und Planung entstanden ist, noch eine Auflistung der unterschiedlichen Programmtests die bei eben dieser Realisierung hilfreich waren und eine Bedienungsanleitung, um die Steuerungsmöglichkeiten bei der Eingabe des Benutzers zu erläutern.

\section{Programmiertagebuch}

Im folgenden Abschnitt sollen die Termine, an denen wir uns als Gruppe im Labor für die Implementierung des Programmcodes für die Hauptaufgabe getroffen haben, einzeln beleuchtet werden und dabei sowohl die Fortschritte als auch unerwartet auftretende Probleme dargestellt werden.\\
Die Lösung der vorbereitenden Übungsaufgabe wird hier nicht näher beschrieben.

\subsection{Vorarbeiten} 

Nach der Veranstaltung zur Aufgabenvorführung in der 22. Kalenderwoche wurde sofort die Planung der Abläufe geplant. Da wir beide noch durch verschiedenen andere Übungen beschäftigt waren, einigten wir uns die Bearbeitung nach Beendigung dieser zu beginnen.\\
Die Übungsaufgabe bearbeiteten wir bereits beim ersten Orientierungstreffen vor der Klausurenphase, bei dem wir auch die Termine der weiteren Bearbeitung vereinbarten, sowie individuelle theoretische Vorarbeiten für die Zeitraum bis zum ersten Treffen. 

\subsection{Erstes Treffen - 1.9.}

\subsection{Zweites Treffen - 8.9.}

\subsection{Drittes Treffen - 15.9.}

\section{Programmtests}

\section{Bedienungsanleitung}

\begin{array}{ll}
	\textbf{Tastenbelegung} & \textbf{Funktion} \\ 
	d/D & Fleischer zur Produktion veranlassen \\ 
	h/H & Grillmeister Wurst auflegen lassen \\ 
	f/F & Fleischer Mineralwasser geben \\ 
	g/G & Grillmeister Mineralwasser geben \\ 
	v/V & Fleischer Mineralwasser entziehen \\ 
	b/B & Grillmeister Mineralwasser entziehen \\ 
	+ & Grilltemperatur erhoehen \\ 
	- & Grilltemperatur reduzieren
\end{array} 